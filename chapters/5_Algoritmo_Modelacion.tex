%qqqqqqqqqqqqqqqqqqqqqqqqqqqqqqqqqqqqqqqqqqqqqqqqqqqqqqqqqqqqqqqqqqqqqqqqq
%Quote
\begin{savequote}[50mm]
%‘‘El cosmos es todo lo que es, todo lo que fue y todo lo que será. Nuestras 
%más ligeras contemplaciones del cosmos nos hacen estremecer: Sentimos como 
%un cosquilleo nos llena los nervios, una voz muda, una ligera sensación como
%de un recuerdo lejano o como si cayéramos desde gran altura. Sabemos que nos
%aproximamos al más grande de los misterios.’’
%\qauthor{Carl Sagan}
\end{savequote}
%qqqqqqqqqqqqqqqqqqqqqqqqqqqqqqqqqqqqqqqqqqqqqqqqqqqqqqqqqqqqqqqqqqqqqqqqq

%*****************************************************
\chapter{Algoritmo y modelación}
\label{cha:Modelo de Spin}
%*****************************************************
%El desarrollo de modelos computacionales permiten reproducir sucesos que escapan a acontecer de una vida humana. Es por eso la importancia de los modelos

En la búsqueda de poder conocer como se pueden alinear los SMBH en AGNs con su entorno, se hace necesario el uso de modelos computacionales que sean capaces de reproducir lo fenómenos físicos que ocurren en el universo. Para nuestro propósito se hace uso de una serie de simulaciones cosmológicas auto-consistentes, que son capaces de simular la evolución de los espines de los SMBHs, ratas de acreción de BHs, feadback de BHs y propiedades de galaxias donde se hospedan los BHs. La simulación que se va usar para tal propósito es el código de Magneto-hidrodinámica para N-cuerpos AREPO \cite{springel2010}, que es capaz de reproducir mallas que cambian de forma, moviendosen con el fluido, y hace seguimiento a las propiedades del gas. Sin embargo, también se hace uso de otros métodos computacionales que permiten conocer cómo se distribuye la materia en el universo o mejor aun, permiten hacer una clasificación y estructuración del mismo. El método que se usa para la clasificación de las estructuras en el universo es el T-web \cite{hahn2007}, que es capaz de categorizar las regiones del universo usando un modelo de sistemas dinámicos. Para el modelo de detección de sub estructuras, que permite localizar zonas de sobre densidad, que indican la presencia de un halo de materia oscura, grupo o cúmulo  de galaxias. 

%*****************************************************
\section{Código Arepo}
\label{sec: codigo arepo}
%***********************************************


%*****************************************************
\section{Caracterización del entorno}
\label{sec: Caracterizacion entorno}
%*****************************************************

%----------------------------------------------------    
    \subsection{Método T-web}
    \label{subsec: Metodo_T-web}
%----------------------------------------------------



%*****************************************************
\section{Método de detección de sub estructuras}
\label{sec: detección sub-estructuras}
%*****************************************************

%----------------------------------------------------    
    \subsection{Método de FoF}
    \label{subsec: FoF}
%----------------------------------------------------


%----------------------------------------------------    
    \subsection{Algorimo Subgrid}
    \label{subsec: Algoritmo subgrid}
%----------------------------------------------------


















%***********************************************************************