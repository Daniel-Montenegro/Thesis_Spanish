%qqqqqqqqqqqqqqqqqqqqqqqqqqqqqqqqqqqqqqqqqqqqqqqqqqqqqqqqqqqqqqqqqqqqqqqqq
%Quote
\begin{savequote}[50mm]
%‘‘El cosmos es todo lo que es, todo lo que fue y todo lo que será. Nuestras 
%más ligeras contemplaciones del cosmos nos hacen estremecer: Sentimos como 
%un cosquilleo nos llena los nervios, una voz muda, una ligera sensación como
%de un recuerdo lejano o como si cayéramos desde gran altura. Sabemos que nos
%aproximamos al más grande de los misterios.’’
%\qauthor{Carl Sagan}
\end{savequote}
%qqqqqqqqqqqqqqqqqqqqqqqqqqqqqqqqqqqqqqqqqqqqqqqqqqqqqqqqqqqqqqqqqqqqqqqqq

%*****************************************************
\chapter{Algoritmo y modelación}
\label{cha:Modelo de Spin}
%*****************************************************
El desarrollo de modelos computacionales permiten reproducir sucesos que escapan a acontecer de una vida humana. Es por eso la importancia de los modelos

En la búsqueda de poder conocer como evoluciona el espín del SMBH, se hace necesario del uso de una serie de simulaciones cosmológicas auto-consistentes, que haciendo uso de la rata de acrección de masa del BH

%*****************************************************
\section{Código Arepo}
\label{sec: codigo arepo}
%***********************************************


%*****************************************************
\section{Caracterización del entorno}
\label{sec: Caracterizacion entorno}
%*****************************************************

%----------------------------------------------------    
    \subsection{Método T-web}
    \label{subsec: Metodo_T-web}
%----------------------------------------------------



%*****************************************************
\section{Método de detección de sub estructuras}
\label{sec: detección sub-estructuras}
%*****************************************************

%----------------------------------------------------    
    \subsection{Método de FoF}
    \label{subsec: FoF}
%----------------------------------------------------


%----------------------------------------------------    
    \subsection{Algorimo Subgrid}
    \label{subsec: Algoritmo subgrid}
%----------------------------------------------------


















%***********************************************************************