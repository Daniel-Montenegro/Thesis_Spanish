%qqqqqqqqqqqqqqqqqqqqqqqqqqqqqqqqqqqqqqqqqqqqqqqqqqqqqqqqqqqqqqqqqqqqqqqqq
%Quote
\begin{savequote}[50mm]
%‘‘El cosmos es todo lo que es, todo lo que fue y todo lo que será. Nuestras 
%más ligeras contemplaciones del cosmos nos hacen estremecer: Sentimos como 
%un cosquilleo nos llena los nervios, una voz muda, una ligera sensación como
%de un recuerdo lejano o como si cayéramos desde gran altura. Sabemos que nos
%aproximamos al más grande de los misterios.’’
%\qauthor{Carl Sagan}
\end{savequote}
%qqqqqqqqqqqqqqqqqqqqqqqqqqqqqqqqqqqqqqqqqqqqqqqqqqqqqqqqqqqqqqqqqqqqqqqqq

%*****************************************************
\chapter{Algoritmo y modelación}
\label{cha:Modelo de Spin}
%*****************************************************
%El desarrollo de modelos computacionales permiten reproducir sucesos que escapan a acontecer de una vida humana. Es por eso la importancia de los modelos

En la búsqueda de poder conocer sobre el alineamiento de los SMBH en AGNs con su entorno, se hace necesario el uso de modelos computacionales que sean capaces de reproducir lo fenómenos físicos que ocurren en este sistema. Para nuestro propósito se hace uso de una serie de simulaciones cosmológicas auto-consistentes, que son capaces de simular la evolución de los espines de los SMBHs, ratas de acreción de BHs, feadback de BHs y entre otras propiedades de las galaxias. La simulación que se va usar para tal propósito es el código Magneto-hidrodinámico para N-cuerpos AREPO \cite{springel2010}, capaz de reproducir mallas que cambian de forma y se mueven con el fluido a medida que corre la simulación,  haciendo seguimiento a las propiedades del gas. Sin embargo, también se hace uso de otros métodos computacionales que permiten conocer cómo se distribuye la materia en el universo o mejor aun, permiten hacer una clasificación y estructuración del mismo. El método usado para la clasificación de estructuras en el universo es el  T-web \cite{hahn2007}, capaz de categorizar las regiones del universo usando un modelo de sistemas dinámicos. Para el modelo de detección de sub estructuras se hace uso del código Arepo, que emplea el método de subfind, este permite localizar zonas de sobre densidad, indicando la presencia de halos de materia oscura, grupo o cúmulo  de galaxias. 

%*****************************************************
\section{Código Arepo}
\label{sec: codigo arepo}
%***********************************************

%El código {\textsc{Arepo}} es una simulación cosmológica, que pretende diseñar una nueva forma de hacer simulaciones hidrodinámicas que son capaces reproducir la dinámica del universo, evolución estelar, formación de estructuras y evolución de parámetros en las galaxia. Este modelo hidrodinámico tiene como objetivo mejorar dos códigos muy usados en la simulaciones de formación y evolución: Mallado de Euleriano y la técnica  hidrodinámica de suavizado de partículas Lagrangiana (SPH). 

En la astrofísica se hace necesario poder contrastar los modelos teóricos con lo observacional. Por eso se emplea los modelos computacionales que parten de  modelos teóricos y buscan poder reproducir los resultados observacionales. Una forma de poder reproducir los procesos que ocurre en el universo, es considerando la dinámica de fluidos, y con ello los modelos hidrodinámicos. Los modelos más sobresalientes en este ámbito son: SPH \cite{monaghan1992} y hidrodinámica Euleriana basada en mallas \cite{stone2008} con refinamiento adaptativo de las mallas (AMR). A pesar de ser los modelos más usados en la hidrodinámica presentan una serie de falencias:

- SPH es pobre en resolución,  ofrecen precisión de bajo orden para el tratamiento de discontinuidades de contacto, y en algunos casos suprime inestabilidades de los fluidos.

- Mallas Eulerianas no producen invariantes galileanos, implicando que los resultados son sensibles a grandes cambios en velocidades de volumen.\\

{\it{Arepo}} toma lo mejor de cada simulación y mejora sus falencias. En general el código {\it{Arepo}} \cite{springel2010} introduce una nueva forma de modelación hidrodinámica continua de mallas dinámicas. El código se describe como \cite{springel2010}:

La precisión del modelo viene determinado mayormente por su nueva forma de diseñar las mallas (grid). Se define la malla bajo la teselación de Voronoi de un conjunto de puntos que se distribuyen espacialmente, permitiendo un movimiento libre. 

El método de Voronoi es el método de interpolación más simple, que se basa en la distancia euclidea. Permite crear celdas de tal forma que la región encerrada esta lo más cerca posible a sus celdas vecinas. Los polígonos se diseñan al unir los puntos entre si, trazando luego una mediatriz entre los segmentos de unión. La intersección de las mediatrices dan como resultado los polígonos y por lo tanto la malla (ver figura \ref{fig: voronoi}). Cada celda en la malla lleva consigo variables de fluido que se conservan (masa, momentum y energía total).
%
\begin{center}
\includegraphics[scale=.35]{./figures/5_Algoritmo_Modelacion/voronoi.png}
\figcaption{\emph{Representación gráfica de la teselación de Voronoi. En esta gráfica se puede ver la doble topología del digrama de Voronoi, pues es equivalente topológicamente a la teselación de Delaunay.}}\label{fig: voronoi}
\end{center}
%
La importancia de la simulación Voronoid, es que permite que los puntos en la malla se muevan con el fluido a la velocidad del fluido, esto garantiza que a medida que trascurren los pasos numéricos se actualice la información del fluido en la celda. Este método posibilita la solucionar el problema de no invariabilidad galileana que emergía del método de la malla de Euler. Al solucionar esto se impide la alta sensibilidad en el cambio de las velocidades, haciendo que la dinámica del sistema sea consistente, y por ende se obtengan valores no sesgados para la alineación de los BHs en cada celda y del momentum angular del gas que circula alrededor del BH. 

Este método permite que en las regiones donde hay una mayor sobre densidad de materia, se genere una mayor cantidad de celdas, proporcionando una modelación más precisa dando como resultado una mejor respuesta a la dinámica del fluido.
%aca voy
El objetivo principal de la simulación es poder solucionar las ecuaciones del fluido de manera muy precisa, para ello hace uso de las ecuaciones de Euler, que son leyes de la conservación de la masa, energía y momentum, que toma la forma de un sistema de ecuaciones diferenciales parciales hiperbólicas. \cite{springel2010} reescribe de forma compacta las ecuaciones de conservación o de fluido de la siguiente forma:

Se introduce el vector de estado    
\begin{align}
    \vec{\bf{U}}= \begin{pmatrix} \rho \\ \rho\vec{v} \\ \rho e \end{pmatrix} = 
    \begin{pmatrix} \rho \\ \rho\vec{v} \\ \rho u+\frac{1}{2}\rho\vec{v}^{2} \end{pmatrix}\,,
\end{align}
donde $\rho$ es la densidad del fluido, $\vec{v}$ es el campo de velocidad y $e=u+ \vec{v}^{2}/2$ es la energía por unidad de masa. el parámetro $u$ indica la energía térmica por unidad de masa. Las cantidades del fluido son dependientes del espacio y del tiempo $\vec{\bf{U}}(\vec{\bf{x}},t)$, se define además la función de flujo

\begin{align}
    \vec{\bf{F}}(\vec{\bf{U}})=
    \begin{pmatrix} \rho\vec{v} \\ \rho\vec{v}\vec{v}^{T}+ P \\ (\rho e + P)\vec{v} \end{pmatrix}\,,
\end{align}
donde $P$ es la ecuación de estado queda la presión del fluido. Es por tanto que la ecuación de Euler se puede escribir de forma compacta como 
\begin{align}
    \frac{\partial\vec{\bf{U}}}{\partial t} + \vec{\nabla}\cdot \vec{\bf{F}}\,.
\end{align}

En el código de {\it{Arepo}} se considera una discretización del fluido como un conjunto de volúmenes finitos, donde cada celda se identifica bajo el uso de un contador "$i$". Cada celda contiene la información de la masa $m_{i}$, momentum $p_{i}$ y la energía $E_{i}$. %
\begin{align}
    {\bf{Q}_{i}} = \begin{pmatrix} m_{i} \\ p_{i} \\ E_{i} \end{pmatrix} = \int_{V_{i}} \vec{\bf{U}}d\vec{\bf{V}}\,.
\end{align}
%
Usando la ecuación de Euler, es posible calcular la razón de cambio de ${\bf{Q}_{i}}$ en el tiempo, al usar el teorema de Gauss
%\
\begin{align}
    \frac{d{\bf{Q_{i}}}}{dt} = -\int_{\partial V_{i}}[\vec{\bf{F}}(\vec{\bf{U}})-\vec{\bf{U}}\vec{\bf{w}}^{T}]d\vec{\bf{n}}\,,
\end{align}
donde $\vec{\bf{w}}$ es la velocidad particular de cada partícula y $\vec{\bf{n}}$ es el vector normal a la superficie.

El nuevo esquema presentado por \cite{springel2010} indica cuales son los pasos (algoritmia) que sigue el código. Parte del estado del fluido para cada celda ${\bf{Q_{i}}}$ y al hacerlo evolucionar es capaz de dar los resultados de no invarianza galileana y buena resolución. 

%*****************************************************
\section{Caracterización del entorno}
\label{sec: Caracterizacion entorno}
%*****************************************************

Al recordar lo mencionado en la sección (\ref{sec: Estructure_Formation}), es posible decir que las simulaciones computacionales asumiendo el régimen lineal o no lineal son capaces de reproducir la estructura del universo. Al obtener la estructura del universo se observa la emergencia de regiones donde hay un cambio en la densidades (una abundancia de materia), donde la estructura cambia considerablemente.

%----------------------------------------------------    
    \subsection{Método T-web}
    \label{subsec: Metodo_T-web}
%----------------------------------------------------
Al considerar el método propuesto por \cite{hahn2007} es posible argumentar lo siguiente: Asumiendo la teoría concerniente a los sistemas dinámicos, se realiza un análisis de estabilidad local para orbitas de prueba alrededor de halos materia oscura. Donde cada una de las partículas de prueba se mueven por acción de un potencial gravitacional $\phi$.

La ecuación de movimiento que describe la partícula en coordenadas comóviles está dada por 
\begin{align}
    \ddot{x}=-\nabla\phi\,.
\end{align}
%
Al asumir que el potencial gravitacional $\phi$ está actuando en el centro de masa ($\bar{x}_{i}$) como un máximo local, da como resultado lo siguiente:
%
\begin{align}
    \nabla\phi(\bar{x}_{i})=0\,.
\end{align}
%
Con esto es posible linealizar la ecuación de movimiento en los puntos donde es máximo ($\bar{x}_{i}$)
%
\begin{align}
    \ddot{x}_{i}=-T_{ij}({\bf{\bar{x}}_{k}})(x_{j}-\bar{x}_{k,j})\,,
\end{align}
%
donde $T_{ij}$ es el tensor de marea dado por el Hessiano del potencial gravitacional
%
\begin{align}
    T_{ij}=\frac{\partial \phi}{\partial r_{i} \partial r_{j}}\,.
\end{align}
%
Este tensor de marea puede ser representado por una matriz real simétrica 3x3, con autovalores $\lambda_{1}>\lambda_{2}>\lambda_{3}$ y  autovectores $\vec{e}_{1}\, \vec{e}_{2}\, \vec{e}_{3}$. Los autovalores son de gran importancia a la hora de clasificar el entorno cosmológico, estos son los indicadores de la estabilidad de la orbita de las partículas de prueba, que se mueven en la dirección del autovector \cite{padmanabhan1995}, \cite{hahn2007}.

Al usar la teoría de \cite{zeldovich1970}, que se estudio en la sección (\ref{subsubsec:Zeldovich_Aproximation}), es posible definir una clasificación a partir de los autovalores en cada región del espacio:\\

$i)$ {\it{vacíos (voids)}: región del espacio donde los tres autovalores $\lambda$ son negativos (orbita estable), indicando una divergencia o expansión en esa región.}\\

$ii)$ {\it{hoja (sheet)}: región donde hay un autovalor positivo $\lambda_{1}>0$ y dos negativos $\lambda_{2} \leq \lambda_{3}< 0$, esto indica un colapso en una dirección del espacio mientras que en las otras dos hay una expansión.}\\

$iii)$ {\it{filamentos (filament)}: región del espacio donde solo hay un autovalor negativo $\lambda_{3}\leq 0$, indicando que hay un colapso en dos direcciones y expansión en una.} \\

$iv)$ {\it{nudo (knot)}: región donde los tres autovalores son positivos, dando como resultado un colapso en las tres direcciones, formando una región de convergencia.}\\

En la figura (\ref{fig: formacion de estructura con autovalores}), es posible observar los cuatro casos donde se representan los tipos de estructuras dependiendo del autovalor.
%
\begin{center}
\includegraphics[scale=.39]{./figures/5_Algoritmo_Modelacion/formacion_estructuras.png}
\figcaption{\emph{El valor de los autovalores permiten clasificar el tipo de estructura en la cual se encuentran \cite{bustamente01}.}}\label{fig: formacion de estructura con autovalores}
\end{center}
%
Lo destacable de este método es su análisis meramente dinámico, lo cual le da más jerarquía que los métodos que usan principios meramentes geométricos, permitiendo identificar zonas de igual densidad pero con propiedades dinámicas de estabilidad diferente. La desventaja de este procedimiento es que depende altamente de los mínimos locales, que solo son justificados en el interior de halos de materia oscura, careciendo de resolución y precisión para lugares diferente a los halos.

El código que se esta implementando para la clasificación del entorno, hace uso del valor "límite" $\lambda_{th}$ para los autovalores \cite{bustamante2015}, pasando de ser $\lambda_{th}=0$ a $\lambda_{th}=0.265$. Se hace uso de este valor límite, porque es capaz de traza de muy buena forma la red cósmica. 

PONER IMAGENES DEL ENTORNO HECHAS POR MI .


%*****************************************************
\section{Método de detección de sub estructuras}
\label{sec: detección sub-estructuras}
%*****************************************************

Una vez se obtiene la clasificación del entorno a gran escala (red cósmica), es necesario poder conocer cómo se distribuye la materia en el interior de estas estructuras, por esto se hace necesario un criterio para clasificación de sub-estructuras. La continuidad de la distribución de materia en el universo implica un grave problema a la hora de caracterizar la materia en dichas sub-estructuras. Poder discretizar la distribución de materia en el espacio posibilita hacer uso de modelos computacionales, capaces de reproducir las estructuras internas.

%----------------------------------------------------    
    \subsection{Método de FoF}
    \label{subsec: FoF}
%----------------------------------------------------

Es uno de los métodos más usados para la detección de sub-estructururas en simulaciones cosmológicas, FoF proviene de las siglas {\it{friend of friend}}, debido a la comparación entre vecinos o "amigos". Es usado en gran medida para la comparación de modelos,  determinando que tan confiable es el método contrastado. 

El método FoF consiste básicamente encontrar intercesiones entre volúmenes definidos alrededor de partículas (ver figura  \ref{fig: FoF esquema}) . La región dada por el volumen es llamada región de vinculación, las estructuras serán caracterizadas por la cantidad de regiones vinculadas. La región de vinculación se definen como esferas de radio $R_{i}$ concéntricas a la partícula, dado por la expresión \cite{bustamente01}:
%
\begin{align}
    R_{i}=\frac{1}{2}b\ell\,,
\end{align}
%
donde b es el parámetro de vinculación y $\ell$ camino libre medio de las partículas en la simulación. 
%
\begin{center}
\includegraphics[scale=.35]{./figures/5_Algoritmo_Modelacion/FoF_metodo.png}
\figcaption{\emph{Esquema de cómo funciona el método de FoF, consiste en determinar que partículas interactuan entre sí a partir de la intersección de regiones de vinculación. }}\label{fig: FoF esquema}
\end{center}
%

%----------------------------------------------------    
    \subsection{Algorimo Subgrid}
    \label{subsec: Algoritmo subgrid}
%----------------------------------------------------

Subfind es un método usado para extraer de forma más precisa las sub-estructuras en las simulaciones cosmológicas. Esta herramienta es más precisa que FoF, sin embargo el uso de los dos método permiten optimizar la caracterización de la estructura. El primer paso que realiza el modelo {\it{subfind}} es calcular la densidad en todo el espacio, a partir de ello se determinan las regiones de sobre densidad (abundancias de densidad), una vez se obtiene las regiones de sobre densidad se procede a construir esferoides concéntricos al lugar donde ocurre la sobre densidad. Cuando la región albergada dentro del esferoide es equivalente a aproximadamente 200 veces la densidad media del universo, que en otras palabras es equivalente al equilibrio virial, se detiene la construcción de esferoides y de determina una sub-estructura \cite{springel2018}.

La mejor forma de optimizar es encontrar el pico de densidad con el método de FoF y a partir de allí usar {\it{subfind}} para determinar la sub-estructura.

~\\

Haciendo uso de Arepo se extraen dos simulaciones {\it{Cosmo01}} y {\it{Cosmo02}}, estas simulaciones tienen  consigo una serie de parámetros de sobre las galaxias, entre los parámetros que más nos interesa son el espín del BH, momentum angular del halo, momentum angular del disco, masa del BH, masa del halo, masa de la galaxia, masa estelar que contiene el halo. ...












%***********************************************************************