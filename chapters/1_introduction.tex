%qqqqqqqqqqqqqqqqqqqqqqqqqqqqqqqqqqqqqqqqqqqqqqqqqqqqqqqqqqqqqqqqqqqqqqqqq
%Quote
%\begin{savequote}[50mm]
%‘‘Equipado con sus cinco sentidos, el hombre explora el universo alrededor 
%suyo y llama a esta aventura Ciencia’’ 
%\qauthor{Edwin Hubble}
%\end{savequote}
%*************************************************************************




%#########################################################################
\chapter{Introducción}
\label{cha:Introduction}

Las observaciones del universo a gran escala, nos inunda con una gran serie de dudas, dudas sobre el origen, constitución y del lugar que ocupamos en este inmenso universo. Nuestra necesidad en encontrar respuesta no ha puesto donde estamos hoy en día, nos ha permitido desarrollar una ciencia que pretende encontrar respuestas, respuestas a tantas dudas. En la actualidad las observaciones a gran escala nos ha permitido evidenciar una estructura, una distribución de materia que abarca todo el universo, estructura que funciona como un conducto por donde se desplaza la materia. Ahora, ¿ este flujo de materia altera la dirección de los objetos dentro de él ?
 
%-----------------------------------------
\section{Preliminares }
\label{sec: prelimenares}
%----------------------------------------

El actual desarrollo de la ciencia nos permite obtener una cantidad absurda de información,  información que es de gran importancia la hora de poder comprender el funcionamiento del universo que nos rodea. Es entonces tarea de la ciencia poder entender cómo funcionan y por qué ocurren dichos eventos. En búsqueda de ello se ha desarrollado una serie de teorías, que van en dirección de explicar por qué y cómo se desarrollan las estructuras del universo, y reconocer los procesos que ocurren en él, procesos que son de gran importancia para el entendimiento de los procesos de formación de galaxias. 
 

 
 

%*************************************************************************