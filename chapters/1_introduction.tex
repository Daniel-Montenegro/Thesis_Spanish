%qqqqqqqqqqqqqqqqqqqqqqqqqqqqqqqqqqqqqqqqqqqqqqqqqqqqqqqqqqqqqqqqqqqqqqqqq
%Quote
\begin{savequote}[60mm]
``Así como los ojos están formados para la astronomía, los oídos lo están para percibir los movimientos de la armonía."
\qauthor{Platón}
\end{savequote}
%*************************************************************************




%#########################################################################
\chapter{Introducción}
\label{cha:Introduction}

Las observaciones del universo a gran escala nos inundan con una gran serie de dudas sobre el origen, constitución y el lugar que ocupamos en este inmenso Universo. Nuestra necesidad de encontrar respuesta nos ha puesto en el lugar donde estamos, nos ha permitido desarrollar una ciencia que pretende encontrar respuestas, respuestas a tantas dudas. En la actualidad las observaciones a gran escala nos ha permitido evidenciar una organización, una distribución de materia que abarca todo el universo, constituida por una innumerable aglomeración de galaxias. Esta estructura  funciona además como conductos por donde fluye materia. Ahora, ¿ este flujo de materia altera la orientación de los objetos dentro de él ?
 
%-----------------------------------------
\section{Preliminares }
\label{sec: prelimenares}
%----------------------------------------
%El actual desarrollo de la ciencia nos permite obtener una cantidad absurda de información,  información que es de gran importancia a la hora de poder comprender el funcionamiento del universo. Es entonces tarea de la ciencia poder entender cómo funciona y por qué ocurren dichos eventos. En búsqueda de ello se ha desarrollado una serie de teorías, que van en dirección de explicar por qué y cómo se desarrollan las estructuras del universo, y reconocer los procesos que ocurren en él, procesos que son de gran importancia para el entendimiento de los procesos de formación y evolución en las galaxias.  
 
El posible alineamiento entre el flujo de materia que circula a través del espacio por regiones de sobre densidad, constituidas por galaxias, pueden proporcionar gran información para entender los procesos de formación de galaxias. En este trabajo se pretende determinar si existe alguna relación entre el espín de Núcleos Activos de Galaxias (AGNs) con su entorno a gran escala. En la actualidad se ha desarrollado estudios observacionales que investigan la evolución de AGNs en diferentes regiones del espacio. Estos estudios han dado como resultado que el plano de polarización de la luz proveniente del AGN está alineado con la estructura local del Universo. A su vez, la dirección del plano de polarización está asociada con la orientación del espín del AGN \cite{hutsemekers2014}. Sin embargo debido a los impedimentos observacionales, donde solo es posible tomar un instante de tiempo, se hace necesario un modelo computacional que permita ver lo que ocurre en varios instantes, y con ello entender en gran medida los procesos evolutivos del AGN, en especial su orientación. Los resultados acá obtenidos tendrían consecuencias interesantes para el entendimiento de los procesos de formación de las galaxias, debido a que el espín del AGN está íntimamente relacionado con procesos a muy grandes escalas, como formación de radio jets y ejección de masa por feedback del AGN, los cuales determinan propiedades fundamentales y globales de la galaxia anfitrión. 

%La teoría de formación de estructuras construidas a partir de las relatividad general y la cosmología, permiten reconstruir las observaciones, sin embargo se hace necesario realizar una serie de suposiciones. 

Las observaciones del universo a gran escala dejan ver una enmarañado sistema, una estructura que da cuenta de la distribución de materia y dinámica de la misma. La teoría de formación de estructuras \cite{zeldovich1970} permite reconstruir la forma del universo observable y proporciona un entendimiento de los sucesos que dan cabida a esta formación. Como ya se ha dicho, estas estructuras funcionan como conductos por donde la materia fluye. Estas estructuras son a su vez una constitución de galaxias que se agrupan bajo la acción de un potencial gravitacional, esta idea permite plantear un modelo de clasificación que faculte la distinción de entornos y poder estudiar la dinámica al interior de cada uno. Usando la teoría de sistemas dinámicos \cite{hahn2007}, se desarrolla una clasificación que parte de tensor de deformación del cual se extrae los autovalores ($\lambda_{1}, \, \lambda_{1}, \, \lambda_{1}$), los cuales proporcionan una criterio clasificación meramente dinámico, además se extrae los autovectores ($\vec{\bf{e_1}}$, $\vec{\bf{e_2}}$, $\vec{\bf{e_3}}$), que dan información de la orientación del flujo del campo de densidad. \cite{forero2009} desarrolla un método de clasificación para la red cósmica T-Web, que permite clasificar cada punto del espacio en los cuatros posibles entornos: voids, sheet, filament y clusters. 

La influencia del flujo de materia en el alineamiento del AGN solo es la primera parte del procesos de alineamiento. El modelo de espín estudia la evolución y alineamiento de los AGNs con procesos internos de la galaxia. \cite{fanidakis2011} desarrolla un modelo de evolución de espín, el cual  muestra la profunda relación entre la evolución del espín de BH con la tasa de acreción de materia y la fusión de Agujeros Negros (BH). Sin embargo en este trabajo se va hacer uso de un modelo de evolución planteado por \cite{Bustamante2018b}. El cual incluye regímenes donde ocurren procesos de acreción caótica, esto al suponer modelos de auto-gravedad alrededor del disco de acreción del BH. 
 
Es entonces el propósito de este trabajo determinar a partir de simulaciones cosmológicas hidrodinámicas (Illustris TNG AGN+ physics model) \cite{springel2010}, y un modelo de evolución de espín de Agujeros Súper Masivos (SMBH) en AGNs, posibles relaciones entre la orientación del AGN y el entorno cosmológico al cual pertenece. Para  determinar la posible relación, se calcula el valor del ángulo entre el espín de los BHs ${\bf{J_{bh}}}$ y el autovector $\vec{\bf{e_3}}$ correspondiente a cada BH.

Este escrito se organiza de la siguiente manera. las secciones 2 y 3 abarca el marco teórico, en la sección 2 se abarca lo correspondiente a la Cosmología y la formación de estructura, la sección 3 a la teoría de los AGNs. La teoría usada para la evolución del espín se encuentra consignada en la sección \ref{cha:Modelo de Spin}. La explicación de los métodos y simulaciones usadas para la obtención de la evolución del espín del AGN y la clasificación de entorno, son presentados en la sección \ref{cha: Algoritmo y modelacion}. Por último los resultados más relevantes, los propósitos a futuro y la conclusiones son presentadas en la sección \ref{cha:cosmic_web}.

%*************************************************************************