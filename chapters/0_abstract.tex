%#########################################################################
%ABSTRACT


\begin{abstracts}
\selectlanguage{spanish}

%Los actuales desarrollos en los modelos de estructura del Universo a gran escala, observaciones y las \'ultimas simulaciones con un poder de resoluci\'on mayor, a su vez muestran un Universo distribuido en regiones (estructuras). Estas estructuras  se distribuyen de una manera compleja y en forma de filamentos. Dichos filamentos representan una sobre densidad de materia y son producto de la no linealidad del sistema. La estructura que conforman el universo se le denomina la red c\'osmica. \\

%La estructura del Universo plantea una duda que es el problema que se pretende resolver. ¿ Es posible encontrar una relaci\'on de las estructuras del Universo con las galaxias, o para ser m\'as específico con el spin de la misma?  Este trabajo nace la categorizaci\'on del mismo Universo, la particularidad de ciertas regiones da indicios de la relaci\'on entre entorno  galaxia; es por esto que se quiere conocer si la "mejor" relaci\'on es por parte del spin de las galaxias.

%La presencia(existencia) de las regiones(estructuras) son resultado de la evoluci\'on del universo. \\

%En estas regiones de sobre densidad se agrupan gran cantidad de materia que est\'a constituida por c\'umulos o agrupaciones de galaxias. Como su nombre lo dice, los c\'umulos o agrupaciones de galaxias son agrupaciones de galaxias debido a la interacci\'on gravitacional entre los cuerpos. Además, se conoce  la existencia de los agujeros negros al interior de las galaxias, el cual tiene asociado un spin. 

%La motivaci\'on de este trabajo es poder encontrar una relaci\'on entre  \\

%%%%%%%%%%%%%%%les  %%%%%%%%%%%%%%%%%%%%%%%%%%%%%%%%%%% un posible relaión entre el entorno ca%%%%%%%%%Las actuaobservaciones han venido demostrando
Los actuales estudios y observaciones han mostrado una posible relación entre la evolución de los AGNs y el entorno cosmológico al cual pertenecen. En dirección de poder encontrar esta relación desde el punto de vista teórico, este trabajo estudia la relación entre la orientación el espín del BH  hospedado en el AGN y la dirección del gradiente del campo de densidad del entorno. Para esto se usa un nuevo modelo de evolución de espín del BH, que incorpora además del régimen de acreción de gas coherente, usado en gran parte de las simulaciones actuales, el régimen de acreción de gas caótica, que representa de una manera general la evolución del espín del BH. Se realizan dos simulaciones magneto hidrodinámicas, en una se considera el régimen de acreción de gas caótico y en la otra el régimen de acreción de gas coherente, las dos con alta resolución y a un  $z=0$. 

Usando la relación entre la orientación del espín del BH ${\bf{J}_{bh}}$ y la orientación del autovector ${\bf{\vec{e}}_{3}}$, se encontró que la relación entre estos dos vectores $\cos\theta$ presenta una dependencia fuerte con la masa del BH. Se muestra que a medida que la masa del BH aumenta, $\cos\theta \to 0$, indicando una ortogonalidad entre el espín del BH y el gradiente del campo de densidad en función de la masa del BH. 







\end{abstracts}


%#########################################################################
